\documentclass{article}
\usepackage[utf8]{inputenc}
\usepackage[T1]{fontenc}
\usepackage[top=1.5cm, bottom=1.5cm, left=1.5cm, right=1.5cm]{geometry}
\usepackage{hyperref}
\usepackage{authblk}
\author[1,2]{Aldo Sayeg Pasos Trejo}
\affil[1]{\textit{Physics Departament. Facultad de Ciencias. Universidad Nacional Autonoma de Mexico}}
\affil[2]{\textit{Visiting Student Researcher for the Berkeley Energy and Climate Institute at University of California, Berkeley}}
\date{August 12, 2016}
\title{Creation of fuels-related input tables for SWITCH Mexico Model}
\begin{document}
\maketitle
\section{Introduction}
This report describes the creation of the "fuel\_costs" TAB file that SWITCH needs an as input. This TAB file consists on quantifying the price of the fuels used in every load area of the SEN ("Sistema eléctrico nacional", National electric system) both in the present and future projections. We used two different sources and methods to create this data file, that resulted in two different .tab files located inside the "MainTabs" folder of the "switch\_mexico\_data" directory.
\\
\\For an easier redaction and understanding  of this document, we must first make an important term distinction. When we refer to a "fuel", we are referring to a particular type of fuel made in a specific location within Mexico (for example: "Carbon Rio Escondido","Diesel Peninsula"). On the other hand, when we refer to a "type of fuel" we refer to a set of fuels that are of the same type (for example, "natural\_gas","coal")
\section{First method: Balancing areas}
The Python script corresponding to this method of file creation is called "SWITCH input creation.py". It is inside the "fuels" folder of the "Costs" directory of the "switch\_mexico\_data" repository.
\\
\\The first method to create the "fuel\_cost" .tab file used  the tables  "BalancingAreas.csv", "Fuels.csv" and "FuelsAnalysis.xlsx" as data reference. These tables are made with information from the PRODESEN ("Programa de desarrollo del sistema eléctrico nacional", Development program for the national electric system)\cite{prodesen}, specifically, from chapter 4\cite{prodesen4}, and from the PIIRCE ("Programa indicativo de instalaciones y retiro de centrales eléctricas", Program for Power Plant Installation and Decommission) Excel files containing fuel costs projections\cite{piircef}. From "Fuels.csv" we extract a list of all the diferent types of fuel that are used in Mexico's electricity generation plants.
\\
\\After that, we extract a table for each of the fuel types from the "FuelsAnalysis.xlsx" file. Here we have, for each type of fuel, a table of yearly-prices, from 2016 to 2030. Also, it points out the balancing area to which the county that produces that fuel belongs to. It is important that although a county is associated with a fuel (and hence, to a balancing area) this might not mean that the fuel is being produced there, but also that it could be the main source point for further distribution afterwards.
\\
\\We then proceed to calculate the cost of every type of fuel for a given load area using the following procedure: from the "BalancingAreas.csv" file, we know which is the balancing area of every load area of the SEN. We can now point out that there are two cases for a given load area and type of fuel:
\begin{itemize}
\item \textbf{There exists at least one county that produces that type of fuel in the balacing area of that load area.} If this happens, the Python script calculates, for every year, the average of the costs of the fuels from the counties that produce that type of fuel and that are in the balancing area of the given load area. Then, the script assigns this value to the yearly cost of that type of fuel at that load area.
\item \textbf{None of the counties produces that type of fuel in the balacing area of that load area.} If this happens, the Python script calculates, for every year, the average of the cost of all of the counties throughout Mexico that produce fuel of that type. Then it assigns this value to the yearly cost of that type of fuel in that load area.
\end{itemize}
%it would be a good idea to introduce a figure here to improve understanding of these method
\section{Second method: Plants of each load zone}
For the second method we used different fuel-prices sources from the ones we used in the first method. We used information from the PIIRCE regarding the list of all of Mexico's power plants\cite{piirceg} and the fuel costs projection \cite{piircef}. The Python script corresponding to this method of file creation is called "SWITCH input creation piirce.py"
\\
\\The "tables" folder of the "fuels" directory of the "switch\_mexico\_data" repository, a file called "PiirceGenerationUsingFuels.csv" contains a list of all of the Mexico's power plants that use fuel to generate electricity. For each of the plants, it specifies the balancing area and load zone to which they contribute. It also specifies the type of fuel and fuel that the generation plant uses. The Python script also imports the "PiirceFuels.csv" file, which consists of the fuel costs projections made by PIIRCE\cite{piircef}.
\\
\\From that information, the Python script creates the "fuels\_cost\_load\_areas\_detailed.tab" file on the "MainTabs" directory. The Python script checks first, for a given load area, the types of fuel that the power plants that contribute to the load area uses. After that, it determinates, for each type of fuel, the fuels used. For every year and for a given fuel type used in that load area, the script assigns the price of that fuel type as the average of the prices of all the fuels of that fuel type that are used in power plants that contribute that to that load area. 
\\
\\One particular thing to note is that this average is not a weighted average. For example, if in the "01\-hermosillo" load area, there are 7 plants that use natural gas as type of fuel and, from the set of power plants that use natural gas, 6 use the "g\_i\_noro" type and 1 uses the "g\_central"  kind. For each year, the price of the natural gas for the "01\-hermosillo" load area will be the average of the price of "g\_i\_noro" and "g\_central" for that year. It won't regard that the number of plants using "g\_i\_noro" is much bigger than the ones using "g\_central". This could be implemented in the future for higher resolution.
\section{Number of fuels types for each load zone}
The .TAB files created with each of the methods have a fundamental difference regarding the number of type of fuels used in each load area.
\\
\\The first method calculates a cost for all of the fuel types inside the load zone, without constraints regarding if the fuel type is actually used by a power plant that contributes to that load area. As a result of this, this method makes a distinction in the types of fuel between the domestic and imported. For example, there is a "domestic natural gas" fuel type and an "imported natural gas" type.
\\
\\In contrast, the second method just calculates the costs of the fuel types used by the power plants that contribute to that load area. As a result of that, there isn't a cost of every type of fuel for each load aera. Also, as in the PIIRCE table none of Mexico's power plants use imported fuels, there is no distinction between domestic and imported fuel types, all of the fuel types are considered domestic.
\begin{thebibliography}{9}
\bibitem{prodesen}
PRODESEN report. \url{http://www.gob.mx/cms/uploads/attachment/file/54139/PRODESEN_FINAL_INTEGRADO_04_agosto_Indice_OK.pdf}
\bibitem{prodesen4}
PRODESEN chapter 4 tables. \url{base.energia.gob.mx/prodesen/PRODESEN-Capitulo4.xlsx}
\bibitem{piircef}
PIIRCE table of fuel prices projection and fuel type relation. \url{http://base.energia.gob.mx/prodesen/BasedeDatos_PIIRCE-2016-2030_PCombustibles.xlsx}
\bibitem{piirceg}
PIIRCE table of Mexico's power plants. \url{http://base.energia.gob.mx/prodesen/BasedeDatos_PIIRCE-2016-2030_Generacion.xlsx}
\end{thebibliography}
\end{document}