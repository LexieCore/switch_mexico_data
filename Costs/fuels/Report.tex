\documentclass{article}
\usepackage[utf8]{inputenc}
\usepackage[T1]{fontenc}
\usepackage[top=1.5cm, bottom=1.5cm, left=1.5cm, right=1.5cm]{geometry}

\author{Aldo Sayeg Pasos Trejo. Facultad de Ciencias, Universidad Nacional Autónoma de México.}
\date{\today}
\title{Creation of fuels-related input tables for SWITCH Mexico Model at University of California, Berkeley}
\begin{document}
\maketitle
\section{Introduction}
This report is about to give details regardig the creation of the "fuel\_costs" TAB file that SWITCH needs an as input. This TAB file consists on the price of the fuels used in every load area of the SEN ("Sistema eléctrico nacional", National electric system). We used two different sources and methods to create this data file, that resulted in two different TAB regarding this subject inside the "MainTabs" folder of the "switch\_mexico\_data" directory.
\\
\\For further reference, we must first make an important termn disctintion. When in this text we talk about a "fuel", we are refering to a particular type of fuel made in an specific location inside of Mexico (for example: "Carbon Rio Escondido","Diesel Peninsula"). When we talk of a "type of fuel" we refer to a category of fuels that are of the same king (for example, "natural_gas","coal")
\section{First method: Balancing areas}
The Python script corresponding to this method of creation is called "SWITCH input creation and SQL import.py". It is inside the "fuels" folder of the "Costs" directory of the "switch\_mexico\_data" repository.
\\
\\The first method to create the fuel\_cost tab used as data reference the tables called "BalancingAreas.csv","Fuels.csv","FuelsAnalysis.xlsx" These tables are made with information from the PRODESEN ("Programa de desarrollo del sistema eléctrico nacional", Development program for the national electric system), especifically, from the chapter 4. From "Fuels.csv" we extract a list of all the diferent types of fuel that are used in Mexico's electricity generation plants.
\\
\\After that, we extract a table for each of the fuel types from the "FuelsAnalysis.xlsx" file. Here we have, for each type of fuel, a table of yearly-prices, from 2016 to 2030, of each one of the fuels of that type. Also, it points out the balancing area to which the county that produces that fuel belongs.
\\
\\Then we proceed to calculate the cost of every type of fuel for a given load area using the following proceedure: from the "BalancingAreas.csv" file, we know which is the balancing area of every load area of the SEN. We can now point out that there are two cases for a given load_area and type of fuel:
\begin{itemize}
\item \textbf{There exists at least one county that produces that type of fuel in the balacing area of that load_area.} If this happens, the Python scrpit calculates, for every year, the average of the costs of the fuels from the counties that produce that type of fuel and that are in the balancing area of the giben load area. Then, the script assigns these value to the yearly cost of that type of fuel at that load area.
\item \textbf{None of the counties produces that type of fuel in the balacing area of that load_area.} If this happens, the Python scrpit calculates, for every year, the average of the cost of all of the counties that produce fuel of that type, withouth consideration of the balacing area to which these fuel-producing counties belong. Then it assigns these value to the yearly cost of that type of fuel in that load area.
\end{itemize}
You can consult figure 1 for further example of these method.
\section{Second method: plants of each load zone}
For the second method
\end{document}